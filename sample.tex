%%%%%%%%%%%%%%%%%%%%%%%%%%%%%%%%%%%%%%%%%%%%%%
% To select a journal, use its code for the 
% journal= option in the \documentclass command.
% The journal codes for this template are:
%
% Antimicrobial Stewardship and Healthcare Epidemiology: ash
%%%%%%%%%%%%%%%%%%%%%%%%%%%%%%%%%%%%%%%%%%%%%%
\documentclass[
  journal=largetwo,
  manuscript=article-type,
  year=2020,
  volume=37,
]{cup-journal}

\usepackage{amsmath}
\usepackage[nopatch]{microtype}
\usepackage{booktabs}

\title{Integrating Artificial Intelligence and Structural Biology to Advance Understanding of G Protein-Coupled Receptor Activation Mechanisms}

\author{A. Khaleq}
\email[A. Khaleq]{ayman@varosync.com}

\author{H. Kabodha}
\email[Varosync]{partnerships@varosync.com}


\addbibresource{gpcr_references.bib}

\keywords{G protein-coupled receptors, artificial intelligence, protein structure prediction, biased signaling, allosteric modulation, AlphaFold, molecular dynamics, drug discovery} 

\begin{document}

\begin{abstract}
G protein-coupled receptors (GPCRs) represent the largest family of membrane proteins in the human genome and serve as critical mediators of cellular signaling. Understanding their activation mechanisms has been fundamentally transformed by recent advances in both structural biology and artificial intelligence-based structure prediction. This review examines the convergence of experimental structural determination methods with computational approaches, including AlphaFold 3, Chai-1, Boltz-1, and related frameworks, to elucidate GPCR conformational dynamics, allosteric regulation, and biased signaling mechanisms. We analyze how these integrated approaches are revolutionizing our understanding of ligand binding, receptor activation, and G protein or arrestin coupling selectivity. The synthesis of AI-powered structure prediction with traditional molecular dynamics simulations provides unprecedented insights into cryptic pocket formation, conformational ensembles, and the molecular basis of functional selectivity. These advances hold significant implications for structure-based drug discovery, particularly in designing biased ligands and allosteric modulators with improved therapeutic profiles.
\end{abstract}

\section{Introduction}

G protein-coupled receptors constitute the largest and most diverse family of membrane receptors in the human genome, with approximately 800 members mediating cellular responses to hormones, neurotransmitters, lipids, photons, and other extracellular stimuli \autocite{Zhang2024GPCR}. These seven-transmembrane domain proteins play essential roles in virtually all physiological processes and represent the molecular targets of approximately one-third of currently approved pharmaceuticals. The therapeutic importance of GPCRs stems from their central position in signal transduction pathways that regulate cardiovascular function, neurological processes, metabolic homeostasis, immune responses, and sensory perception.

The fundamental challenge in GPCR biology has been understanding how these receptors transduce extracellular signals across the plasma membrane to activate intracellular effector proteins. This process involves complex conformational changes that span approximately 40 angstroms from the extracellular ligand-binding site to the intracellular G protein or arrestin coupling interface \autocite{Weis2018}. The allosteric nature of GPCR signaling, whereby ligand binding at one site promotes structural rearrangements that facilitate effector coupling at a distant site, has made mechanistic understanding particularly challenging. Moreover, the discovery that different ligands can stabilize distinct receptor conformations that preferentially activate specific downstream pathways (termed biased signaling or functional selectivity) has added additional layers of complexity to our understanding of GPCR pharmacology \autocite{Wootten2024}.

Recent technological advances have revolutionized our ability to study GPCR structure and dynamics. The development of protein engineering strategies to stabilize active receptor states, combined with innovations in X-ray crystallography and particularly cryo-electron microscopy, has enabled determination of high-resolution structures of GPCRs in complex with diverse ligands and signaling partners \autocite{Hauser2021}. These structural insights have been further enhanced by spectroscopic techniques including nuclear magnetic resonance spectroscopy and fluorescence-based methods that can probe conformational dynamics in solution. Parallel to these experimental advances, computational approaches have undergone a transformation through the application of deep learning to protein structure prediction.

The introduction of AlphaFold 2 in 2020 marked a watershed moment in structural biology, demonstrating that deep neural networks trained on experimental protein structures and evolutionary information could predict three-dimensional protein structures with near-experimental accuracy \autocite{Jumper2021}. This breakthrough has been extended to more complex challenges relevant to GPCR biology, including prediction of protein-protein interactions, protein-ligand complexes, and incorporation of post-translational modifications. Recent iterations including AlphaFold 3 \autocite{Abramson2024}, along with alternative frameworks such as Chai-1 \autocite{ChaiDiscovery2024}, Boltz-1 \autocite{Wohlwend2024}, and their derivatives including HelixFold3 \autocite{Liu2024HelixFold3} and Protenix \autocite{ByteDance2025}, have further expanded the scope of predictable biomolecular structures to encompass nucleic acids, small molecules, ions, and covalent modifications. These advances are particularly relevant for GPCRs given their interactions with diverse ligands and their propensity to form oligomeric complexes.

The convergence of experimental structural determination with AI-based prediction methods, enhanced by molecular dynamics simulations and machine learning analyses of conformational ensembles, is providing unprecedented insights into GPCR activation mechanisms. This integration enables investigation of questions that were previously intractable, including the structural basis of ligand bias, mechanisms of allosteric modulation through cryptic pockets, and the conformational landscapes underlying functional selectivity. Understanding these mechanisms at atomic resolution has direct implications for rational drug design, particularly in developing biased agonists that selectively activate beneficial signaling pathways while avoiding adverse effects mediated by alternative pathways.

This review examines the current state of knowledge regarding GPCR activation mechanisms, with emphasis on insights derived from the integration of experimental and computational structural biology approaches. We discuss advances in understanding conformational dynamics, allosteric regulation, biased signaling, and the application of these insights to drug discovery. Particular attention is given to recent developments in AI-based structure prediction and their impact on our ability to model GPCR conformational states and ligand interactions.

\section{Literature Review}

\subsection{Structural Advances in GPCR Biology}

The past two decades have witnessed remarkable progress in determining high-resolution structures of GPCRs. The first GPCR structure, that of bovine rhodopsin, was reported in 2000 and revealed the canonical seven-transmembrane helix architecture that defines this receptor family \autocite{Palczewski2000}. However, progress in determining structures of ligand-activated GPCRs remained limited until the development of specialized protein engineering approaches. Key innovations included the use of T4 lysozyme or other fusion partners to facilitate crystallization, thermostabilizing mutations to increase receptor stability, and nanobody-based strategies to stabilize active conformations. These technical advances enabled determination of structures representing both inactive and active receptor states, providing the first direct structural insights into activation mechanisms.

The advent of cryo-electron microscopy as a routine structural biology tool has been particularly transformative for GPCR structural studies. Unlike X-ray crystallography, which requires protein crystallization and often necessitates extensive protein engineering, cryo-EM can determine structures of native or near-native receptors in complex with physiological binding partners. This has enabled determination of structures of GPCR-G protein and GPCR-arrestin complexes at high resolution, revealing the detailed molecular interactions underlying signal transduction \autocite{Zhang2024GPCR}. For class B GPCRs, cryo-EM structures have elucidated the two-domain binding mechanism whereby peptide hormones engage both the extracellular domain and the transmembrane core \autocite{Zhang2017GLP1}. Notable examples include structures of the GLP-1 receptor in complex with heterotrimeric Gs protein, which have provided mechanistic insights into receptor activation and revealed the binding mode of orally active small-molecule agonists \autocite{Kawai2020}.

Structural studies have revealed that GPCR activation involves concerted conformational changes across multiple conserved structural motifs \autocite{Hauser2021}. Key molecular switches include the DRY motif at the intracellular end of transmembrane helix 3, the NPxxY motif in helix 7, and the PIF motif connecting helices 3 and 6. Upon agonist binding, these motifs undergo rearrangements that are coupled to an outward movement of the intracellular portion of helix 6, creating the binding interface for G proteins or arrestins. However, the extent and nature of these conformational changes vary across GPCR classes and even among different receptors within the same class, reflecting the functional diversity of this receptor family.

\subsection{Conformational Dynamics and Allosteric Mechanisms}

While static structures have provided invaluable snapshots of GPCR conformations, understanding receptor function requires knowledge of the dynamic equilibria between different conformational states. Nuclear magnetic resonance spectroscopy and fluorescence-based single-molecule studies have revealed that GPCRs exist in multiple conformational states even in the absence of ligands, with ligand binding shifting the population distribution rather than inducing new conformations de novo \autocite{Weis2018}. This conformational selection model of allostery implies that full agonists, partial agonists, antagonists, and inverse agonists differ in their ability to stabilize specific conformational ensembles, which in turn determines their functional effects.

Molecular dynamics simulations have been instrumental in characterizing GPCR conformational landscapes and the pathways connecting different functional states. Markov state models constructed from extensive simulation data have revealed the existence of multiple metastable intermediate states along activation pathways, with different ligands influencing the kinetics of transitions between these states \autocite{Plattner2015}. These computational approaches have been particularly valuable for understanding cryptic pocket formation, whereby transient binding sites emerge through conformational fluctuations \autocite{Meller2023}. Such cryptic pockets represent attractive targets for allosteric modulator development, as they may exhibit greater subtype selectivity than orthosteric sites \autocite{Hollingsworth2019}.

Recent studies employing Gaussian accelerated molecular dynamics combined with deep learning approaches have provided comprehensive maps of conformational changes in GPCRs upon binding of allosteric modulators \autocite{Do2023}. These investigations have demonstrated that positive allosteric modulators and negative allosteric modulators exert their effects by restricting receptor conformational dynamics, confining receptors to specific functional states. The cooperative binding of orthosteric and allosteric ligands is mediated through long-range conformational coupling that involves both the transmembrane domain and intracellular regions \autocite{Faure2022}. Understanding these allosteric mechanisms at atomic detail provides a foundation for rational design of modulators with desired pharmacological properties.

\subsection{Biased Signaling and Functional Selectivity}

A paradigm shift in GPCR pharmacology emerged with the recognition that different ligands binding to the same receptor can selectively activate distinct downstream signaling pathways. This phenomenon, termed biased signaling or functional selectivity, has profound implications for drug development \autocite{Fan2025,Wootten2024}. Traditional GPCR drugs were developed based on their ability to activate or inhibit the primary signaling pathway, typically G protein-mediated responses. However, GPCRs can also signal through beta-arrestin-mediated pathways, which were initially thought to primarily mediate receptor desensitization and internalization but are now recognized as important signaling mechanisms in their own right.

The structural basis of biased signaling is beginning to emerge from comparative structural and biophysical studies. Biased ligands appear to stabilize receptor conformations that differ subtly but critically from those stabilized by balanced agonists \autocite{Zhao2023}. These conformational differences are particularly evident in the intracellular regions of the receptor that interface with G proteins or arrestins, explaining how ligand-induced changes in the orthosteric pocket can be transmitted to determine effector coupling selectivity. Studies of the mu-opioid receptor have been particularly instructive, as G protein-biased agonists hold promise for developing analgesics with reduced respiratory depression and other adverse effects mediated by arrestin signaling \autocite{Fan2025}.

The mechanisms underlying biased signaling involve multiple factors including ligand-specific conformational changes, temporal profiles of receptor activation, and cell type-specific differences in the relative abundance of signaling components. Kinetic context has emerged as an important determinant of signaling bias, with the residence time of ligands and the temporal dynamics of receptor conformational changes influencing the balance between different signaling pathways \autocite{Caniceiro2025}. Understanding these multifaceted mechanisms requires integration of structural data with kinetic measurements and cellular signaling assays, highlighting the need for multidisciplinary approaches to GPCR pharmacology.

\subsection{Artificial Intelligence in Protein Structure Prediction}

The application of deep learning to protein structure prediction has undergone rapid evolution. AlphaFold 2 demonstrated that neural networks incorporating attention mechanisms and leveraging multiple sequence alignments could predict protein structures with unprecedented accuracy \autocite{Jumper2021}. The architecture combines an Evoformer module that processes evolutionary and geometric information with a structure module that iteratively refines predicted coordinates. This approach achieved median accuracy comparable to experimental methods on benchmark datasets, representing a major advance over previous computational methods.

Extensions of the AlphaFold framework to predict protein complexes (AlphaFold-Multimer) and more recently to predict structures of biomolecular assemblies including ligands, nucleic acids, and modifications (AlphaFold 3) have expanded the applicability of AI-based prediction to increasingly complex biological systems relevant to drug discovery \autocite{Abramson2024}. AlphaFold 3 employs a diffusion-based architecture that directly predicts atomic coordinates, replacing the frame-based approach of AlphaFold 2. This enables modeling of arbitrary chemical structures beyond the standard amino acids, although prediction accuracy for protein-ligand interactions remains lower than for protein-only predictions.

Alternative deep learning frameworks including Chai-1 \autocite{ChaiDiscovery2024}, Boltz-1 \autocite{Wohlwend2024}, HelixFold3 \autocite{Liu2024HelixFold3}, and Protenix \autocite{ByteDance2025} have been developed in parallel, each with distinct architectural features and training strategies. Chai-1 incorporates protein language model embeddings \autocite{Lin2023} to enable strong single-sequence predictions and introduces constraint features to incorporate experimental data such as cross-linking mass spectrometry results. Boltz-1 and its successor Boltz-2 \autocite{Boltz2025} focus on both structure prediction and binding affinity estimation, demonstrating improved performance on ligand docking benchmarks. Systematic benchmarking using datasets such as FoldBench \autocite{Xu2025} and PoseBusters has revealed that while these methods achieve impressive accuracy on many targets, challenges remain for specific interaction types including antibody-antigen complexes and protein-ligand interactions involving large conformational changes. Tools such as ABCFold \autocite{Morehead2024} facilitate comparative analysis of these methods.

\subsection{Integrating AI Prediction with Molecular Dynamics}

While AI-based structure prediction provides static models, GPCR function inherently involves conformational dynamics. Integration of predicted structures with molecular dynamics simulations offers a powerful approach to explore receptor behavior. Predicted structures can serve as starting points for simulations exploring conformational changes, ligand binding pathways, and allosteric communication \autocite{Do2023}. Enhanced sampling methods including Gaussian accelerated molecular dynamics, metadynamics, and replica exchange approaches enable efficient exploration of conformational space beyond the timescales accessible to conventional simulations.

Machine learning approaches are increasingly being applied to analyze molecular dynamics trajectories and extract mechanistic insights \autocite{Moon2022,Mao2025}. Deep learning models trained on simulation data can identify collective variables describing conformational changes, predict binding free energies, and classify receptor states. The combination of AI-predicted structures with physics-based simulations provides a complementary approach wherein predicted structures are refined and validated through dynamics, while simulations provide training data to improve prediction models \autocite{AupiC2024}. This bidirectional integration is particularly valuable for GPCRs given their conformational complexity.

Recent work has demonstrated that predicted structures from AlphaFold can capture cryptic pocket opening in certain cases, although the extent varies depending on the system \autocite{Meller2023}. When partial pocket opening is predicted, simulations launched from these structures can more efficiently sample full opening compared to simulations starting from closed experimental structures. This suggests a promising workflow wherein multiple predicted conformations are generated through manipulation of multiple sequence alignments or other input perturbations, followed by targeted simulations to explore functionally relevant conformational changes.

\subsection{Applications to GPCR Drug Discovery}

The integration of AI-based structure prediction with experimental structural biology is transforming structure-based drug discovery for GPCRs \autocite{Schauperl2022}. High-confidence predicted structures can guide virtual screening campaigns to identify novel ligands, even for receptors lacking experimental structures. Ensemble docking approaches using multiple predicted or simulated conformations can account for receptor flexibility and improve hit rates. For allosteric sites, which often exhibit greater conformational variability than orthosteric sites, ensemble-based approaches are particularly important \autocite{Hollingsworth2019}.

Recent studies have applied these approaches to design ligands targeting neosurfaces, composite interfaces formed between proteins and small molecules \autocite{Marchand2025}. Deep learning models based on learned molecular surface representations can design binders to drug-bound protein complexes, enabling the development of chemically induced protein interactions for therapeutic applications. For GPCRs, such approaches could enable design of modulators that selectively affect interactions with specific G protein subtypes or accessory proteins, providing an additional dimension of selectivity beyond traditional ligand design.

The ability to predict structures of GPCRs in complex with biased ligands, even with moderate accuracy, provides valuable insights for ligand optimization \autocite{Qiao2024}. Physics-informed deep learning models that incorporate knowledge of intermolecular interactions show improved generalization compared to purely data-driven approaches \autocite{Moon2022}. Integration of structure prediction with binding affinity prediction, as exemplified by recent developments in models such as Boltz-2 \autocite{Boltz2025}, enables prioritization of compounds for synthesis and testing. While challenges remain, particularly in accurately predicting the effects of ligand modifications on signaling bias, the rapid pace of methodological development suggests that these limitations may be overcome through continued algorithmic innovation and integration with experimental validation. The design of dynamic proteins with controllable conformational changes \autocite{Aupic2024} represents a frontier for engineering synthetic GPCRs with novel signaling properties.

\section{Conclusion}

The integration of artificial intelligence-based structure prediction with experimental structural biology and molecular dynamics simulations represents a transformative advance in our understanding of GPCR activation mechanisms and our ability to design selective ligands. The convergence of these approaches has illuminated the structural basis of allosteric regulation, biased signaling, and conformational dynamics that were previously inaccessible to experimental or computational methods alone. Key insights include the identification of cryptic pockets as druggable allosteric sites, the elucidation of conformational ensembles underlying functional selectivity, and the atomic-level characterization of biased ligand binding modes.

Despite remarkable progress, significant challenges remain. Accurate prediction of protein-ligand interactions, particularly for systems undergoing large conformational changes, remains less reliable than protein-only predictions. The relationship between ligand binding modes and downstream signaling bias is not yet fully understood at a predictive level. Cell type-specific effects and kinetic factors that influence signaling outcomes in vivo are difficult to capture in structural studies or simulations. Integration of multi-omics data with structural information may be necessary to bridge the gap between molecular mechanisms and physiological effects.

Future progress will likely involve tighter integration of prediction, simulation, and experimental validation, enabled by rapid-cycle design-make-test frameworks and continued algorithmic development. The application of these integrated approaches to understudied GPCR subfamilies, including orphan receptors and receptors implicated in rare diseases, holds promise for expanding the druggable genome. Ultimately, the synergy between artificial intelligence and structural biology is poised to accelerate the discovery of next-generation GPCR therapeutics with improved efficacy and reduced adverse effects.

\paragraph{Acknowledgments}
We are grateful for the technical assistance of contributors.

\paragraph{Funding Statement}
This research was supported by independent research initiatives.

\paragraph{Competing Interests}
The authors declare no competing interests.

\paragraph{Data Availability Statement}
All data referenced in this review are available through their respective published sources.

\paragraph{Ethical Standards}
This review meets all ethical guidelines for academic publishing.

\paragraph{Author Contributions}
Conceptualization: A.K.; H.K. Writing original draft: A.K.; H.K. All authors approved the final submitted draft.

\printendnotes

\defbibnote{preamble}{References cited in this review represent a synthesis of current literature on GPCR mechanisms and AI-based structure prediction.}

\printbibliography[prenote={preamble}]

\end{document}
